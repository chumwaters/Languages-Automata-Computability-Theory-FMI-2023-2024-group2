\documentclass{article}
\usepackage[document]{ragged2e}
\usepackage[utf8]{inputenc}
\usepackage[russian]{babel}
\usepackage{amssymb}
\usepackage{xcolor}
\usepackage{mathtools}
\usepackage{tikz}
\usepackage{amsmath}
\usetikzlibrary{automata,positioning}

\newcommand{\cleft}[2][.]{%
  \begingroup\colorlet{savedleftcolor}{.}%
  \color{#1}\left#2\color{savedleftcolor}%
}
\newcommand{\cright}[2][.]{%
  \color{#1}\right#2\endgroup
}

\begin{document}
\begin{center}
    {\huge Упражнение 3}
\end{center}

\vspace{15pt}

\section{Недетерминирани крайни автомати}
      \textit{Детерминистичността} при детерминираните крайни автомати се изразява в това,
      че във всеки един момент от четенето на подадената му дума, детерминираният
      автомат има едно и точно едно състояние, в което може да отиде с току-що прочетената
      буква. По тази причина, измислянето на ДКА за даден език може да бъде
      тромаво в някои случаи. Например, нека разгледаме следния ДКА, разпознаващ
      езика $L = (\{ab\} \cup \{aba\})^*$. 

      \begin{center}
        \begin{tikzpicture}[shorten >=1pt,node distance=2cm,on grid,auto] 
            \node[state,initial,accepting] (q_0)   {$q_0$}; 
            \node[state] (q_1) [right=of q_0] {$q_1$};
            \node[state, accepting] (q_2) [right=of q_1] {$q_2$}; 
            \node[state, accepting] (q_3) [right=of q_2] {$q_3$};  
            \node[state] (q_4) [below=of q_1] {$q_4$};
            \path[->]
            (q_0) edge   node {$a$} (q_1)
                  edge   node [swap] {$b$} (q_4)
            (q_1) edge   node {$a$} (q_4)
                  edge   node {$b$} (q_2)
            (q_2) edge   node {$a$} (q_3)
                  edge   node {$b$} (q_4)
            (q_3) edge   [bend right, above] node {$a$} (q_1)
                  edge   [bend left, below] node {$b$} (q_2)
            (q_4) edge   [loop below] node {$a$,$b$} ();
        \end{tikzpicture}
    \end{center}

    Може да се покаже, че не съществува ДКА с по-малко от пет състояния за този език,
    така че този автомат е минимален по отношение на броя на състоянията му. Сега нека за момент забравим за това, че функцията
    на преходите е \textit{функция} по същество (тоест за всяка двойка от състояние и буква от домейна
    има точно едно състояние, което ѝ съответства) и да се опитаме да измислим по-прост
    "автомат" \hspace{0.01cm} за дадения език. Най-естествена е следната идея.
    \begin{center}
      \begin{tikzpicture}[shorten >=1pt,node distance=2cm,on grid,auto] 
          \node[state,initial,accepting] (q_0)   {$q_0$}; 
          \node[state] (q_1) [right=of q_0] {$q_1$};
          \node[state] (q_2) [below right=1.5 of q_0, yshift = -0.7cm] {$q_2$}; 
          \path[->]
          (q_0) edge [bend right, below] node {$a$} (q_1)
          (q_1) edge   node {$b$} (q_2)
                edge [bend right, above] node {$b$} (q_0)
          (q_2) edge   node {$a$} (q_0);
      \end{tikzpicture}
    \end{center}

    Наистина, езика на този "автомат" \hspace{0.01cm} е точно исканият. Броят на състоянията е само три,
    а идеята — съвършено проста. Формално обаче, тази илюстрация не съответства на това,
    което досега сме наричали \textit{автомат}. За да разширим понятието, нека
    въведем следната дефиниция.

    \vspace{15pt}

    \textbf{Дефиниция 1.} \textbf{Недетерминиран краен автомат (НКА)} е наредена петорка $N = (Q, \Sigma, \Delta, s, F)$, където \\
    — $Q$ е \textit{крайно} множество от \textbf{състояния}, \\
    — $\Sigma$ е азбука, \\
    — $s \in Q$ е \textbf{началното състояние}, \\
    — $F \subseteq Q$ е множеството от \textbf{крайни състояния} и \\
    — $\Delta$, \textbf{функцията на преходите}, е функция от $Q \times \Sigma$ в $2^Q$.
 
    \vspace{15pt}

    По тази дефиниция, автоматът на последната картинка вече точно съответства на 
    конкретен НКА.
    
    \vspace{5pt}

    Следва да дефинираме \textbf{разширената функция на преходите} \\ $\hat{\Delta} : Q \times \Sigma^* \rightarrow 2^Q$
    за НКА. Отново ще сторим това с индукция по думата-аргумент $w$.

    \vspace{5pt}

    \textbf{База:} $\hat{\Delta}(q,\epsilon) = \{q\}$, за всяко $q \in Q$.

    \vspace{5pt}

    \textbf{Стъпка:} Сега да предположим, че $w = ua$ за някои $u \in \Sigma^*$ и 
    $a \in \Sigma$. Нека също предположим, че $\hat{\Delta}(q,u) = \{q_1,...,q_k\}$. 
    Тогава \\
    \begin{center}
        $\hat{\Delta}(q,w) = \bigcup\limits_{i=1}^k \Delta(q_i,a)$.
    \end{center}

    \vspace{5pt}

    Kазваме, че недетерминираният автомат $N$ \textbf{разпознава} думата $w$, ако просто
    $\hat{\Delta}(s,w) \cap F \neq \varnothing$.

    \vspace{5pt}

    С $L(N)$ означаваме множеството от всички думи $w \in \Sigma^*$, които $N$ разпознава.
    Тоест, $L(N) = \{w \in \Sigma^*$ | $\hat{\Delta}(s,w) \cap F \neq \varnothing$\}.

    \vspace{15pt}

    Вече демонстрирахме удобството, което недетерминираните крайни автомати предоставят.
    Сега ни вълнува въпроса, дали те са неотменна част от теорията на крайните автомати.
    Тоест, дали съществуват езици, които се разпонават от някой НКА, но не се разпознават
    от нито един ДКА. Следната теорема ни дава отговора на този въпрос, а именно — не,
    всичко, което можем да сторим с недетерминиран автомат, можем да сторим и с детерминиран
    такъв.

    \vspace{15pt}

    \textbf{Теорема 1(на Рабин-Скот).} За всеки НКА $N$ съществува ДКА $A$, такъв че
    $L(N) = L(A)$.

    \vspace{15pt}

    Доказателството на тази теорема, макар и да не е изложено тук, е \textit{конструктивно}.
    С други думи, то предоставя \textit{алгоритъм}, по който от даден НКА $N$ да 
    конструираме \textit{еквивалентен} на него ДКА $A$ (тоест такъв, че $L(N) = L(A)$).
    Силно неформално, идеята е следната: \\
    \vspace{15pt}
    Нека ни е даден НКА $N = (Q_N,\Sigma,\Delta,s,F_N)$. Ще конструираме еквивалентен
    на него ДКА $A = (Q_A,\Sigma,\delta,s,F_A)$. \\
    \hspace{1cm} 1. Инициализираме $Q_A$, $\delta$ и $F_A$ съответно с $\varnothing$. \\
    \hspace{1cm} 2. Добавяме $\{s\}$ към $Q_A$. \\
    \hspace{1cm} 3. Разглеждаме състояние $q \in Q_A$, което не е маркирано като \\
    \hspace{1cm} \textit{обходено}.  За всяка буква $a \in \Sigma$ добавяме множеството \\
    \hspace{1cm} $B = \bigcup\limits_{q_i \in q}\Delta(q_i,a)$ към $Q_A$ и в последствие добавяме двойката \\
    \hspace{1cm} ($(q,a),B$) към $\delta$. Маркираме състоянието $q$ като \textit{обходено} и \\
    \hspace{1cm} прилагаме стъпка 3 отново, ако още има необходени състояния. \\
    \hspace{1cm} 4. След като необходените състояния се изчерпат "сканираме" \hspace{0.01cm} \\
    \hspace{1cm} множеството $Q_A$ и добавяме към $F_A$ тези състояния $B$, за които \\
    \hspace{1cm} $B \cap F_N \neq \varnothing$.

    \vspace{15pt}

    \textbf{Пример 1.} Да приложим алгоритъма за детерминиране върху автомата с три 
    състояния, който предложихме за езика $L = ({ab} \cup {aba})^*$.

    \begin{center}
      \begin{tikzpicture}[shorten >=1pt,node distance=2cm,on grid,auto] 
          \node[state,initial,accepting] (q_0)   {$q_0$}; 
          \node[state] (q_1) [right=of q_0] {$q_1$};
          \node[state] (q_2) [below right=1.5 of q_0, yshift = -0.7cm] {$q_2$}; 
          \path[->]
          (q_0) edge [bend right, below] node {$a$} (q_1)
          (q_1) edge   node {$b$} (q_2)
                edge [bend right, above] node {$b$} (q_0)
          (q_2) edge   node {$a$} (q_0);
      \end{tikzpicture}
    \end{center}

    Първо инициализираме $Q$, $\delta$ и $F$ с $\varnothing$. След това добавяме състоянието
    $\{q_0\}$ към $Q$ и преминаваме към стъпка 3. \\

    \vspace{5pt}

    Единственото състояние в $Q$ е $\{q_0\}$ и то не е маркирано като обходено. 
    Добавяме $\{q_1\}$ и $\varnothing$ към $Q$ съответно заради преходите от $q_0$ с
    буквите $a$ и $b$. Добавяме в $\delta$ преходите $((\{q_0\},a),\{q_1\})$ и 
    $((\{q_0\},b),\varnothing)$.\\

    \vspace{5pt}

    Сега разглеждаме необходеното състояние $\{q_1\}$. Добавяме $\{q_0,q_2\}$ в $Q$
    заради преходите от $q_1$ с $b$. Към $\delta$ добавяме преходите
    $((\{q_1\},a),\varnothing)$ и \\
    $((\{q_1\},b),\{q_0,q_2\})$.

    \vspace{5pt}

    Следва да разгледаме необходеното състояние $\varnothing$. Добавяме преходите
    $((\varnothing,a),\varnothing)$ и $((\varnothing,b),\varnothing)$.

    \vspace{5pt}

    Разглеждаме необходеното състояние $\{q_0,q_2\}$. Добавяме $\{q_0,q_1\}$ към
    $Q$ заради преходите от $q_0$ и $q_1$ с буквата $a$. Добавяме и преходите \\ 
    $((\{q_0,q_2\},a),\{q_0,q_1\})$ и $((\{q_0,q_2\},b),\varnothing)$ към $\delta$.

    \vspace{5pt}

    Преминаваме към необходеното състояние $\{q_0,q_1\}$. Добавяме към $\delta$
    преходите $((\{q_0,q_1\},a),\{q_1\})$ и $((\{q_0,q_1\},b),\{q_0,q_2\})$.

    \vspace{5pt}

    С това се изчерпаха необходените състояния. Сканираме $Q$ и заключаваме, че
    $F = \{\{q_0\},\{q_0,q_2\},\{q_0,q_1\}\}$.

    \vspace{5pt}

    Резултатът от детерминирането е изобразен отдолу.

    \begin{center}
      \begin{tikzpicture}[shorten >=1pt,node distance=2cm,on grid,auto] 
          \node[state,initial,accepting] (q_0)   {$\{q_0\}$}; 
          \node[state] (q_1) [right=of q_0] {$\{q_1\}$};
          \node[state, accepting] (q_2) [right=of q_1] {$\{q_0,q_2\}$}; 
          \node[state, accepting] (q_3) [right=of q_2] {$\{q_0,q_1\}$};  
          \node[state] (q_4) [below=of q_1] {$\varnothing$};
          \path[->]
          (q_0) edge   node {$a$} (q_1)
                edge   node [swap] {$b$} (q_4)
          (q_1) edge   node {$a$} (q_4)
                edge   node {$b$} (q_2)
          (q_2) edge   node {$a$} (q_3)
                edge   node {$b$} (q_4)
          (q_3) edge   [bend right, above] node {$a$} (q_1)
                edge   [bend left, below] node {$b$} (q_2)
          (q_4) edge   [loop below] node {$a$,$b$} ();
      \end{tikzpicture}
  \end{center}

  Ясно се вижда, че това всъщност е детерминираният автомат от началото на упражнението,
  но с променени "имена" \hspace{0,01cm} на състоянията.

\vspace{25pt}

\section{Задачи}
    \textbf{Задача 1.} (а) Кои от следните думи се разпознават от автомата показан на картинката долу вляво? \\
    (i) $a$ \\
    (ii) $aa$ \\
    (iii) $aab$ \\
    (iv) $\epsilon$
    
    
      \begin{tikzpicture}[shorten >=1pt,node distance=2cm,on grid,auto] 
          \node[state,initial,accepting] (q_0)   {$q_0$}; 
          \node[state] (q_1) [right=of q_0] {$q_1$};
          \node[state,initial,accepting] (p_0) [right=4cm of q_1] {$q_0$};
          \node[state] (p_1) [right=of p_0] {$q_1$};
          \node[state,accepting] (p_2) [right=of p_1] {$q_2$};
          \node[state] (p_3) [below=of p_1] {$q_3$};
          \path[->]
          (q_0) edge [loop above] node {$a$} ()
                edge node [swap] {$a$} (q_1)
          (q_1) edge [loop right] node {$b$} ()
          (p_0) edge node {$a$} (p_1)
          (p_1) edge node {$b$} (p_2)
                edge node [swap] {$b$} (p_3)
          (p_2) edge [bend left, below] node {$a$} (p_1)
          (p_3) edge [bend right, below] node {$a$} (p_2);
      \end{tikzpicture}

   \vspace{10pt}

   (б) Кои от следните думи се разпознават от автомата показан на картинката горе вдясно? \\
   (i) $\epsilon$ \\
   (ii) $ab$ \\
   (iii) $abab$ \\
   (iv) $aba$ \\
   (v) $abaa$ \\
    
   \vspace{35pt}

   \textbf{Задача 2.} Намерете НКА за всеки от следните езици. \\
   (а) $(\{ab\}^*)(\{ba\}^*)\cup\{a\}\{a\}^*$ \\
   (б) $((\{ab\} \cup \{aab\})^*\{a\}^*)^*$ \\
   (в) $((\{a\}^*\{b\}^*\{a\}^*)^*\{b\})^*$ \\ 
   (г) $(\{ba\} \cup \{b\})^* \cup (\{bb\} \cup \{a\})^*$ 

   \vspace{15pt}
   
   \textbf{Задача 3.} Намерете НКА за всеки от следните езици. След това детерминирайте
   посочените от вас автомати. \\
   (а) $(\{ab\}\cup\{aab\}\cup\{aba\})^*$ \\
   (б) $(\{a\}\cup\{b\})^*\{aabab\}$ \\
   (в) $(\{a\}\cup\{b\})^*\{a\}(\{a\}\cup\{b\})(\{a\}\cup\{b\})(\{a\}\cup\{b\})(\{a\}\cup\{b\})$
\vspace{25pt}

\section{Решения}
  \textbf{Задача 1.}(а) Всички освен $aab$. \\
  (б) Всички освен $abaa$.

  \vspace{15pt}

  \textbf{Задача 2.} \\ (а)\hspace {6cm} (б)\\ 
  \vspace{10pt}
  \begin{tikzpicture}[shorten >=1pt,node distance=2cm,on grid,auto] 
      \node[state,initial,accepting] (q_0)   {$q_0$}; 
      \node[state] (q_1) [above=of q_0] {$q_1$};
      \node[state] (q_2) [right=of q_0] {$q_2$}; 
      \node[state, accepting] (q_3) [right=of q_2] {$q_3$};  
      \node[state, accepting] (q_4) [below=of q_0] {$q_4$};
      \node[state,initial,initial where=right,accepting] (p_0) [right=4cm of q_3] {$q_0$};
      \node[state] (p_1) [above=of p_0] {$q_1$};
      \node[state] (p_2) [below left=of p_0] {$q_2$};
      \node[state] (p_3) [below right=of p_0] {$q_3$};
      \node[state,accepting] (p_4) [left=of p_0] {$q_4$};
      \path[->]
      (q_0) edge [bend left] node {$a$} (q_1)
            edge node {$b$} (q_2)
            edge node [swap] {$a$} (q_4)
      (q_1) edge [bend left] node {$b$} (q_0)
      (q_2) edge node {$a$} (q_3)
      (q_3) edge   [bend left] node {$b$} (q_2)
      (q_4) edge   [loop left] node {$a$} ()
      (p_0) edge [bend left] node {$a$} (p_1)
            edge node  {$a$} (p_2)
            edge node {$a$} (p_4)
      (p_1) edge [bend left] node {$b$} (p_0)
      (p_2) edge node [swap] {$a$} (p_3)
      (p_3) edge node  {$b$} (p_0)
      (p_4) edge [loop above] node {$a$} ()
            edge node {$a$} (p_1)
            edge node [swap] {$a$} (p_2);
  \end{tikzpicture}

\vspace{5pt}

(в) \hspace{6cm} (г)\\
\vspace{5pt}
\begin{tikzpicture}[shorten >=1pt,node distance=1.5cm,on grid,auto] 
      \node[state,initial,accepting] (q_0) {$q_0$};
      \node[state] (q_1) [right=of q_0] {$q_1$};
      \node[state,initial,accepting] (p_0) [right=5cm of q_1] {$q_0$};
      \node[state] (p_1) [above=of p_0] {$q_1$};
      \node[state,accepting] (p_2) [right=3cm of p_0] {$q_2$};
      \node[state] (p_3) [above=of p_2] {$q_3$}; 
      \path[->]
      (q_0) edge [bend left] node {$a$} (q_1)
            edge [loop above] node {$b$} ()
      (q_1) edge [loop above] node {$a$} ()
            edge [bend left] node {$b$} (q_0)
      (p_0) edge [bend left] node {$b$} (p_1)
            edge [loop below] node {$b$} ()
            edge node [swap] {$a$} (p_2)
            edge node {$b$} (p_3)
      (p_1) edge [bend left] node {$a$} (p_0)
      (p_2) edge [bend left] node {$b$} (p_3)
            edge [loop below] node {$a$} ()
      (p_3) edge [bend left] node {$b$} (p_2);
\end{tikzpicture}

\end{document} 