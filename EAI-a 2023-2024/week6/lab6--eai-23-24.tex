\documentclass{article}
\usepackage[document]{ragged2e}
\usepackage[utf8]{inputenc}
\usepackage[russian]{babel}
\usepackage{amssymb}
\usepackage{xcolor}
\usepackage{mathtools}
\usepackage{tikz}
\usepackage{amsmath}
\usepackage{mathrsfs}
\usetikzlibrary{automata,positioning}

\newcommand{\cleft}[2][.]{%
  \begingroup\colorlet{savedleftcolor}{.}%
  \color{#1}\left#2\color{savedleftcolor}%
}
\newcommand{\cright}[2][.]{%
  \color{#1}\right#2\endgroup
}

\newcommand{\bleft}{
    \boldsymbol{\left(\right.}
}

\newcommand{\bright}{
    \boldsymbol{\left.\right)}
}

\newcommand{\bplus}{
    \boldsymbol{+}
}

\newcommand{\bepsilon}{
    \boldsymbol{\epsilon}
}

\begin{document}
\begin{center}
    {\huge Упражнение 6}
\end{center}

\vspace{15pt}

\section{Минимален автомат}
    \textbf{Дефиниция 1.} Нека $L \subseteq \Sigma^*$ е език, и нека $x,y \in \Sigma^*$. 
    Казваме, че $x$ и $y$ са \textbf{еквивалентни по отношение на $\boldsymbol{L}$},
    означено $x \approx_L y$, ако за всяка дума $z \in \Sigma^*$, следното е вярно:
    $xz \in L \iff yz \in L$. Тази релация наричаме \textbf{релация на Нероуд за езика $\boldsymbol{L}$}.

    \vspace{5pt}

    \hspace{15pt} Забележете, че $\approx_L$ е релация на еквивалентност над $\Sigma^*$ за всеки език
    $L$ над $\Sigma$. Класовете на еквивалентност на тази релация, неформално казано,
    са множествата от думи, такива че слепването на коя да е фиксирана дума към края
    на кои да е две от тях
    води до получаването на две думи, за които или и двете са в $L$, или и двете са
    извън $L$. С $[x]_L$ означаваме класът на думата $x$ по отношение на 
    релацията на Нероуд за $L$.  

    \vspace{15pt}
    
    \textbf{Пример 1.} Да намерим класовете на еквивалентност за езика \\
    $L = (ab + ba)^*$. Не е трудно да
    се съобрази, че този език има точно \textit{четири} класа на еквивалентност
    по отношение на релацията на Нероуд, а именно: \\
    (1) $[\epsilon]_L = L$, \\
    (2) $[a]_L = La$, \\
    (3) $[b]_L = Lb$, \\
    (4) $[aa]_L = L(aa + bb)\Sigma^*.$ \\

    \vspace{5pt}

    \hspace{15pt} За (1) можем да отбележим, че не всеки език има свойството, че \\ $[\epsilon]_L = L$.
    Разгледайте например езика $(ab)^*(a+b)$. (2) и (3) са класовете на думите,
    които представляват конкатенация на дума от $L$ с $a$ или $b$ съответно. Тези
    думи могат да се продължат до дума в $L$ \textit{единствено} с представител на $b L^*$ и $a L^*$ съответно. 
    (4) е класът на думите с дължина $\geq 2$, които не са в $L$. Каквото и да 
    конкатенираме към произволни два представителя на този клас, резултатната дума
    ще бъде извън езика.

    \vspace{15pt}

    \textbf{Дефиниция 2.} Нека $A = (Q,\Sigma,\delta,s,F)$ е ДКА. Казваме, че две думи 
    $x,y \in \Sigma^*$ са \textbf{еквивалентни по отношение на $\boldsymbol{A}$}, 
    означено с $x \sim_A y$, ако 
    \begin{center}
        $\hat{\delta}(s,x) = \hat{\delta}(s,y)$.
    \end{center}
    Тази релация наричаме \textbf{релация на Нероуд по автомата $\boldsymbol{A}$}.
    
    \vspace{5pt}

    Отново, $\sim_L$ е релация на еквивалентност над $\Sigma^*$ за всеки език $L$ над
    $\Sigma$. Класовете ѝ на еквивалентност се идентифицират с достижимите от $s$
    състояния в $A$. Означаваме съответстващия на състоянието $q \in Q$ клас с $E_q$.

    \vspace{45pt}

    \textbf{Пример 2.} Да разгледаме отново езика $L = (ab + ba)^*$. Следният автомат
    $A$ разпознава точно $L$.
    \begin{center}
        \begin{tikzpicture}[shorten >=1pt,node distance=2cm,on grid,auto] 
            \node[state,initial,accepting] (q_1) {$q_1$}; 
            \node[state] (q_2) [right=of q_1] {$q_2$};
            \node[state,accepting] (q_3) [right=of q_2] {$q_3$};
            \node[state] (q_4) [below=of q_1] {$q_4$};
            \node[state] (q_5) [right=of q_4] {$q_5$};
            \node[state] (q_6) [right=of q_5] {$q_6$};
            \path[->]
            (q_1) edge node {$a$} (q_2)
                  edge [bend right,swap] node {$b$} (q_4)
            (q_2) edge node {$a$} (q_5)
                  edge [bend left] node {$b$} (q_3)
            (q_3) edge [bend left] node {$a$} (q_2)
                  edge [bend right,swap] node {$b$} (q_6)
            (q_4) edge [bend right,swap] node {$a$} (q_1)
                  edge node {$b$} (q_5)
            (q_5) edge [loop below] node {$a$,$b$} ()
            (q_6) edge [bend right,swap] node {$a$} (q_3)
                  edge node {$b$} (q_5);
        \end{tikzpicture}
    \end{center}

    Класовете на еквивалентност на $\sim_A$ са \\
    (1) $E_{q_1} = (ba)^*$, \\
    (2) $E_{q_2} = La$, \\
    (3) $E_{q_3} = (ba)^*abL$, \\
    (4) $E_{q_4} = b(ab)^*$, \\
    (5) $E_{q_5} = L(bb + aa)\Sigma^*$, \\
    (6) $E_{q_6} = (ba)^*abLb$.

    \vspace{15pt}

    \hspace{15pt} Връзката между двете релации, които дефинирахме е следната.\\
    \textbf{Теорема 1.} За всеки ДКА $A = (Q,\Sigma,\delta,s,F)$ и всеки две думи
    $x,y \in \Sigma^*$, ако $x \sim_A y$, то $x \approx_{L(A)} y$. \\
    
    \vspace{15pt}

    \hspace{15pt} Друг начин да изкажем тази връзка е да кажем, че релацията $\sim_L$ 
    \textit{прецизира} релацията $\approx_{L(A)}$. В общия случай, за две релации
    на еквивалентност $R$ и $R'$ казваме, че $R$ \textbf{прецизира} $R'$, ако за всеки
    $x,y$ $xRy$ влече $xR'y$. Обратно, ще казваме, че $R'$ \textbf{апроксимира} $R$.
    Ако релацията на еквивалентност $\sim$ прецизира релацията
    на еквивалентност $\approx$, то всеки от класовете на еквивалентност на $\sim$ се
    съдържа в някой от класовете на $\approx$. С други думи, всеки клас на еквивалентност
    на $\approx$ е обединение на един или повече класове на еквивалентност на $\sim$.

    \vspace{15pt}

    \hspace{15pt} \textbf{Теорема 1} ни дава, че всеки автомат разпознаващ език $L$ има
    поне толкова състояния, колкото са класовете на еквивалентност на $\approx_L$.
    Тоест, броят на класовете на еквивалентност на тази релация е \textit{долна граница}
    за броя на състоянията на кой да е автомат, разпознаващ езика $L$. Но дали тази
    долна граница е достижима? Следната теорема ни дава отговор на този въпрос. \\
    
    \vspace{15pt}

    \textbf{Теорема 2(на Майхил-Нероуд).} Нека $L \subseteq \Sigma^*$ е регулярен език.
    Тогава съществува ДКА с точно толкова на брой състояния, колкото е броят на 
    класовете на еквивалентност на $\approx_L$, който разпознава $L$.

    \vspace{15pt}

    \hspace{15pt} Идеята е най-естествена. Използвайки само релацията $\approx_L$ ще 
    конструираме ДКА $A = (Q,\Sigma,\delta,s,F)$, такъв че $L(A) = L$. $A$ е 
    дефиниран както следва:
    \begin{center}
        $Q = \{[w]$ | $w \in \Sigma^*$\}, множеството от класовете на еквивалентност 
        на $\approx_L$. \\
        $s = [\epsilon]$, класът на еквивалентност на $\epsilon$ спрямо $\approx_L$. \\
        $F = \{[w]$ | $w \in L$\}, множествтото от класовете на еквивалентност на думите
        в $L$ спрямо $\approx_L$. \\
        Накрая, за всеки клас $[w] \in Q$ и всяка буква $a \in \Sigma$, дефинираме
        $\delta([w],a) = [wa]$. 
    \end{center}

    \hspace{15pt} Разбира се, такъв автомат може да се конструира само ако $\approx_L$
    има краен брой класове на еквивалентност. Как сме сигурни, че това е така? Езикът 
    $L$ е регулярен. Значи съществува ДКА $A'$, такъв че $L(A') = L$. Според 
    \textbf{Теорема 2} конструираният от нас автомат $A$ също разпознава $L$. Но 
    броя на състоянията на $A$ е точно броят на класовете на еквивалентност на 
    $\approx_L$. Този брой на свой ред е по-малък от броя на класовете на еквивалентност
    на $\sim_{A'}$, съгласно \textbf{Теорема 1}. Сега е достатъчно
    да забележим, че броят на класовете на еквивалентност на $\sim_{A'}$ е точно
    броя на състоянията на $A'$, който е краен. Значи броят на състоянията на $A$ е
    ограничен отгоре от крайно число.

    \vspace{15pt}

    \textbf{Пример 3.} Минималния автомат за езика $L = (ab + ba)^*$ можем да 
    получим от \textbf{Пример 1} и \textbf{Пример 2}. 
    
    \begin{center}
        \begin{tikzpicture}[shorten >=1pt,node distance=2cm,on grid,auto] 
            \node[state,initial,accepting] (q_1) {$[\epsilon]$}; 
            \node[state] (q_2) [right=of q_1] {$[a]$};
            \node[state] (q_3) [below=of q_1] {$[b]$};
            \node[state] (q_4) [right=of q_3] {$[aa]$};
            \path[->]
            (q_1) edge node {$a$} (q_2)
                  edge [bend right,swap] node {$b$} (q_3)
            (q_2) edge node {$a$} (q_4)
                  edge [bend left] node {$b$} (q_1)
            (q_3) edge [bend right,swap] node {$a$} (q_1)
                  edge node {$b$} (q_4)
            (q_4) edge [loop right] node {$a$,$b$} ();
        \end{tikzpicture}
    \end{center}
     
    \vspace {15pt}

    \hspace{15pt} Изведеното до тук не предоставя алгоритъм с който по даден ДКА $A$
    да конструираме минималния автомат за $L(A)$. Следва да опишем един такъв алгоритъм.
    Първо да дефинираме релация 
    $\equiv_A$ $\subseteq$ $Q \times Q$ по следния начин. За всеки две състояния
    $q,p \in Q$
    \begin{center}
        $q \equiv_A p \iff (\forall w \in \Sigma^*)[\hat{\delta}(q,w) \in F \iff \hat{\delta}(p,w) \in F]$.
    \end{center}
    Лесно се забелязва, че тази релация е релация на еквивалентност. Нейните класове
    на еквивалентност са точно тези множества от състояния, които трябва да се "слеят"
    \hspace{0,01cm} в $A$, за да получим минималния автомат за $L(A)$. \\
    \hspace{15pt} Алгоритъмът ни ще трябва да изчислява класовете на еквивалентност на
    $\equiv_A$. За целта дефинираме следната редица от апроксимации на релацията 
    $\equiv_A$. За всеки две състояния $q,p \in Q$ 
    \begin{center}
        $q \equiv_A^n p \iff (\forall w \in \Sigma^*)[|w| \leq n \implies (\hat{\delta}(q,w) \in F \iff \hat{\delta}(p,w) \in F)]$.
    \end{center}
    \hspace{15pt} Очевидно, всяка от релациите $\equiv_A^0,\equiv_A^1,\equiv_A^2,...$
    e апроксимация на предшественика си в редицата. Освен това, $q \equiv_A^0 p$ е вярно
    тстк $q$ и $p$ са едновременно крайни или едновременно некрайни състояния. Тоест
    $\equiv_A^0$ има точно два класа на еквивалентност: $F$ и $Q \setminus F$. Сега
    остана само да покажем как $\equiv_A^n$ зависи от $\equiv_A^{n-1}$ за всяко
    $n \geq 1$. Връзката е следната
    \begin{center}
        $(\forall q \in Q)(\forall p \in Q)[q \equiv_A^n p \iff q \equiv_A^{n-1} p$ \& $(\forall a \in \Sigma)[\delta(q,a) \equiv_A^{n-1} \delta(p,a)]]$.
    \end{center}
    Алгоритъмът за изчисление на $\equiv_A$ има следния вид. \\
    \vspace{5pt}
    \hspace{15pt} (1) По начало класовете на $\equiv_A^0$ са $F$ и $Q \setminus F$; \\
    \hspace{15pt} (2) За всяко $n = 1,2,...$ изчисли класовете на $\equiv_A^n$ чрез класовете на \\
    \hspace{34pt}$\equiv_A^{n-1}$ докато тези на $\equiv_A^n$ не станат същите като тези на
    $\equiv_A^{n-1}$. \\
    \hspace{15pt} (3) Върни $\equiv_A^n$ спрямо текущото достигнатото $n$.

    \vspace{15pt}

    \textbf{Пример 4.} Да приложим алгоритъма върху автомата от
    \textbf{Пример 2}. Очакваме естествено да получим като резултат автомата от 
    \textbf{Пример 3}. \\
    \vspace{10pt}
    По начало, класовете на $\equiv_A^0$ са $\{q_1,q_3\}$ и $\{q_2,q_4,q_5,q_6\}$. \\
    \vspace{10pt}
    След първата итерация на (2), класовете на $\equiv_A^1$ са $\{q_1,q_3\}$, $\{q_2\}$,
    $\{q_4,q_6\}$ и $\{q_5\}$. \\
    \vspace{10pt}
    След втората итерация на (2), класовете не се разбиват допълнително. Алгоритъмът
    терминира и връща горните четири множества като класовете на еквивалентност на 
    $\equiv_A$. Това са именно състоянията на минималния автомат за $L = (ab + ba)^*$.
    Крайни са тези от тях, които са получени от разбиването на класа на крайните 
    състояния (тоест тези, които съдържат само крайни състояния). Преход от състояние
    $Q$ с буквата $\sigma$ има към състоянието $\{p \in Q$ | $\delta(q,\sigma) = p$, за някое $q \in Q$\}.
    Резултатният автомат естествено е \textit{изоморфен} на този от \textbf{Пример 3}.
    
    \begin{center}
        \begin{tikzpicture}[shorten >=1pt,node distance=2cm,on grid,auto] 
            \node[state,initial,accepting] (q_1) {$\{q_1,q_3\}$}; 
            \node[state] (q_2) [right=of q_1] {$\{q_2\}$};
            \node[state] (q_3) [below=of q_1] {$\{q_4,q_6\}$};
            \node[state] (q_4) [right=of q_3] {$\{q_5\}$};
            \path[->]
            (q_1) edge node {$a$} (q_2)
                  edge [bend right,swap] node {$b$} (q_3)
            (q_2) edge node {$a$} (q_4)
                  edge [bend left] node {$b$} (q_1)
            (q_3) edge [bend right,swap] node {$a$} (q_1)
                  edge node {$b$} (q_4)
            (q_4) edge [loop right] node {$a$,$b$} ();
        \end{tikzpicture}
    \end{center}

\section{Задачи}
    \textbf{Задача 1.} (а) Намерете класовете на еквивалентност спрямо $\approx_L$ за 
    всеки от следните езици: \\
    (i) $L = (aab + ab)^*$.\\
    (ii) $L = \{w$ | $w$ има поддумата $aababa$\}.\\
    (iii) $L = \{ww^{rev}$ | $w \in \{a,b\}^*$\}.\\
    (iv) $L = \{ww$ | $w \in \{a,b\}^*\}$\\
    (v) $L_n = (a+b)*a(a+b)^n$, където $n \in N^+$. \\
    
    (б) За тези езици от (а), за които отговорът е краен, дайте минимален ДКА, разпознаващ
    съответния език.



\vspace{25pt}

\section{Решения}
    

\end{document} 