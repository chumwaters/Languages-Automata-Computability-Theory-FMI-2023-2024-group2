\documentclass{article}
\usepackage[document]{ragged2e}
\usepackage[utf8]{inputenc}
\usepackage[russian]{babel}
\usepackage{amssymb}
\usepackage{xcolor}
\usepackage{mathtools}
\usepackage{tikz}
\usepackage{amsmath}
\usetikzlibrary{automata,positioning}

\newcommand{\cleft}[2][.]{%
  \begingroup\colorlet{savedleftcolor}{.}%
  \color{#1}\left#2\color{savedleftcolor}%
}
\newcommand{\cright}[2][.]{%
  \color{#1}\right#2\endgroup
}

\begin{document}
\begin{center}
    {\huge Упражнение 4}
\end{center}

\vspace{15pt}

\section{Затвореност относно регулярните операции}
      В това упражнение целим да покажем, че класът на \textit{автоматните} езици е
      затворен относно някои от операциите, които въведохме върху езици — 
      \textit{обединение, конкатенация} и \textit{звезда на Клини}.

      \vspace{15pt}

      \textbf{Дефиниция 1.} Казваме, че един език $L$ е \textbf{автоматен}, ако
      съществува ДКА $A$, такъв че $L(A) = L$.

      \vspace{15pt}

      \textbf{Теорема 1.} Класът на автоматните езици е затворен относно операциите \\
      (а) \textit{обединение}; \\
      (б) \textit{конкатенация}; \\
      (в) \textit{звезда на Клини}; \\

     \vspace{15pt}

     Доказателството на горната теорема би се състояло в това да се предложат конструкции,
     които по подадени два детерминирани автомата $A_1$ и $A_2$ или единствен ДКА
     $A$ строят нови детерминирани автомати, разпознаващи съответно езиците 
     $L(A_1) \cup L(A_2)$, $L(A_1) \circ L(A_2)$ и $L(A)^*$, след което да се докаже 
     коректността на предложените конструкции. Ние ще се ограничим само до това да 
     предложим конструкции.

     \vspace{15pt}
     
     (а) \textit{Обединение.} Нека $A_1 = (Q_1,\Sigma,\delta_1,s_1,F_1)$ и 
     $A_2 = (Q_2,\Sigma,\delta_2,s_2,F_2)$ са детерминирани крайни автомати. Ще
     конструираме нов НКА $N$, такъв че $L(N) = L(A_1) \cup L(A_2)$. Идеята е да
     добавим ново начално състояние, което недетерминистично да извършва всички преходи,
     които $s_1$ и $s_2$ извършват, ефективно опитвайки се да \textit{познаем} 
     дали входната дума принадлежи на $L(A_1)$ или на $L(A_2)$. \\
     \hspace{15pt} Формално, ако Б.О.О.
     допуснем, че $Q_1 \cap Q_2 = \varnothing$, то $N$ дефинираме както следва:
     $N = (Q,\Sigma,\Delta,s,F)$, където $s \notin Q_1 \cup Q_2$,
     \begin{center}
        $Q = Q_1 \cup Q_2 \cup \{s\}$, \\
     \end{center}

     \begin{center}
     $F = 
     \begin{cases}
      F_1 \cup F_2 \cup \{s\}, & \text { ако } s_1 \in F_1 \text { или } s_2 \in F_2, \\
      F_1 \cup F_2, & \text { иначе },
    \end{cases}$
    \end{center}      
    
     а $\Delta$ дефинираме по следния начин

     \begin{center}
        $\Delta(q,a) =
        \begin{cases}
          \{\delta_1(q,a)\}, & \text { ако } q \in Q_1, \\
          \{\delta_2(q,a)\}, & \text { ако } q \in Q_2, \\
          \{\delta_1(q,a)\} \cup \{\delta_2(q,a)\}, & \text { ако } q = s.
        \end{cases}$
     \end{center}

     \vspace{15pt}

     \textbf{Пример 1.} Ще построим НКА за езика $\{a\} \cup \{ab\}$ използвайки 
     горната конструкция върху следните два автомата за езиците $\{a\}$ и $\{ab\}$
     съответно.

     \vspace{10pt}

    \begin{tikzpicture}[shorten >=1pt,node distance=2cm,on grid,auto] 
      \node[state,initial] (q_0) {$q_0$}; 
      \node[state,accepting] (q_1) [right=of q_0] {$q_1$};
      \node[state] (q_2) [below=of q_0] {$q_2$};
      \node[state,initial] (p_0) [right=3cm of q_1] {$p_0$};
      \node[state] (p_1) [right=of p_0] {$p_1$};
      \node[state,accepting] (p_2) [right=of p_1] {$p_2$};
      \node[state] (p_3) [below=of p_1] {$p_3$};
      \path[->]
      (q_0) edge node {$a$} (q_1)
            edge node {$b$} (q_2)
      (q_1) edge node {$a$,$b$} (q_2)
      (q_2) edge [loop left] node {$a$,$b$} ()
      (p_0) edge node {$a$} (p_1)
            edge node {$b$} (p_3)
      (p_1) edge node {$b$} (p_2)
            edge node {$a$} (p_3)
      (p_2) edge node {$a$,$b$} (p_3)
      (p_3) edge [loop left] node {$a$,$b$} ();
  \end{tikzpicture}

  \vspace{10pt}

  Резултатния НКА е следният. 

  \vspace{10pt}

  \begin{center}
  \begin{tikzpicture}[shorten >=1pt,node distance=2cm,on grid,auto] 
    \node[state,initial] (s) [below left=1.5cm of q_2] {$s$} ;
    \node[state] (q_0) {$q_0$}; 
    \node[state,accepting] (q_1) [right=of q_0] {$q_1$};
    \node[state] (q_2) [below=of q_0] {$q_2$};
    \node[state] (p_0) [below=of q_2] {$p_0$};
    \node[state] (p_1) [right=of p_0] {$p_1$};
    \node[state,accepting] (p_2) [right=of p_1] {$p_2$};
    \node[state] (p_3) [below=of p_1] {$p_3$};
    \path[->]
    (s)   edge [bend right] node [swap] {$a$} (q_1)
          edge [bend left] node [swap] {$a$} (p_1)
          edge node [yshift=-0.1cm] {$b$} (q_2)
          edge [bend right] node [swap] {$b$} (p_3)
    (q_0) edge node {$a$} (q_1)
          edge node [swap] {$b$} (q_2)
    (q_1) edge node [swap] {$a$,$b$} (q_2)
    (q_2) edge [loop left] node {$a$,$b$} ()
    (p_0) edge node {$a$} (p_1)
          edge node {$b$} (p_3)
    (p_1) edge node {$b$} (p_2)
          edge node {$a$} (p_3)
    (p_2) edge node {$a$,$b$}  (p_3)
    (p_3) edge [loop below] node {$a$,$b$} ();
\end{tikzpicture}
\end{center}

\vspace{15pt}

(б) \textit{Конкатенация.} Отново, нека $A_1 = (Q_1,\Sigma,\delta_1,s_1,F_1)$ и \\
$A_2 = (Q_2,\Sigma,\delta_2,s_2,F_2)$, са детерминирани крайни автомати. Конструираме
НКА $N$ такъв че $L(N) = L(A_1) \circ L(A_2)$. Идеята този път е $N$ да симулира $A_1$
докато стигне до някое негово крайно състояние, след което да направи
недетерминистичен "скок" \hspace{0.1cm} към $A_2$ чрез имитация на преходите
на $s_2$ със следващата прочетена буква. \\
\hspace{15pt} Формално, ако Б.О.О.
допуснем, че $Q_1 \cap Q_2 = \varnothing$, то $N$ дефинираме както следва:
     $N = (Q,\Sigma,\Delta,s_1,F)$,
     \begin{center}
        $Q = Q_1 \cup Q_2$ \\
     \end{center}

     \begin{center}
      $F = 
      \begin{cases}
       F_1 \cup F_2, & \text { ако } s_2 \in F_2, \\
       F_2, & \text { иначе },
      \end{cases}$
     \end{center} 
      
     а $\Delta$ дефинираме по следния начин

     \begin{center}
        $\Delta(q,a) =
        \begin{cases}
          \{\delta_1(q,a)\}, & \text { ако } q \in Q_1 \setminus F_1, \\
          \{\delta_1(q,a)\} \cup \{\delta_2(s_2,a)\}, & \text { ако } q \in F_1, \\
          \{\delta_2(q,a)\}, & \text { ако } q \in Q_2.
        \end{cases}$
     \end{center}


\vspace{15pt}

\textbf{Пример 2.} Ще построим НКА за езика $\{a\} \circ \{ab\}$ използвайки 
горната конструкция върху автоматите от \textbf{Пример 1}.
\vspace{25pt}

\begin{center}
      \begin{tikzpicture}[shorten >=1pt,node distance=2cm,on grid,auto] 
        \node[state,initial] (q_0) {$q_0$}; 
        \node[state] (q_1) [right=of q_0] {$q_1$};
        \node[state] (q_2) [below=of q_0] {$q_2$};
        \node[state] (p_0) [right=3cm of q_1] {$p_0$};
        \node[state] (p_1) [right=of p_0] {$p_1$};
        \node[state,accepting] (p_2) [right=of p_1] {$p_2$};
        \node[state] (p_3) [below=of p_1] {$p_3$};
        \path[->]
        (q_0) edge node {$a$} (q_1)
              edge node [swap] {$b$} (q_2)
        (q_1) edge node [swap] {$a$,$b$} (q_2)
              edge [bend left] node {$a$} (p_1)
              edge [bend right] node [swap] {$b$} (p_3)
        (q_2) edge [loop left] node {$a$,$b$} ()
        (p_0) edge node {$a$} (p_1)
              edge node {$b$} (p_3)
        (p_1) edge node {$b$} (p_2)
              edge node {$a$} (p_3)
        (p_2) edge node {$a$,$b$}  (p_3)
        (p_3) edge [loop below] node {$a$,$b$} ();
    \end{tikzpicture}
    \end{center}

\vspace{15pt}

(в) \textit{Звезда на Клини.} Нека $A = (Q,\Sigma,\delta,s,F)$ е ДКА. Ще конструираме
НКА $N$, такъв че $L(N) = L(A)^*$. Естествената идея би била да конкатенираме 
$A$ към самия себе си без да създаваме копие и да направим началното състояние крайно, ако вече не е 
такова. Този подход обаче е грешен, защото може да вкара нови думи в езика в случаите,
в които $A$ има преходи обратно към началното състояние.
За да получим автомат за $L(A)^*$ трябва да добавим ново начално и заключително състояние,
което да имитира преходите на оригиналното начално състояние вместо да правим оригиналното
начално състояние крайно. \\

\hspace{15pt} Формално $N$ дефинираме както следва:
     $N = (Q',\Sigma,\Delta,s',F')$,
     \begin{center}
      $s' \notin Q$ \\
      $Q' = Q \cup \{s'\}$ \\
      $F' = F \cup \{s'\}$,
     \end{center} 
      
     а $\Delta$ дефинираме по следния начин

     \begin{center}
        $\Delta(q,a) =
        \begin{cases}
          \{\delta(q,a)\}, & \text { ако } q \in Q \setminus F_1, \\
          \{\delta(s,a)\} \cup \{\delta(q,a)\}, & \text { ако } q \in F_1, \\
          \{\delta(s,a)\}, & \text { ако } q = s'.
        \end{cases}$
     \end{center}

\vspace{15pt}

\textbf{Пример 3.} Ще построим НКА за езика $\{ab\}^*$ използвайки 
горната конструкция върху автомата за $\{ab\}$ от \textbf{Пример 1}.

\vspace{15pt}

\begin{center}
      \begin{tikzpicture}[shorten >=1pt,node distance=2cm,on grid,auto] 
        \node[state,initial,accepting] (s') {$s'$}; 
        \node[state] [right=of s'] (p_0) {$p_0$};
        \node[state] (p_1) [right=of p_0] {$p_1$};
        \node[state,accepting] (p_2) [right=of p_1] {$p_2$};
        \node[state] (p_3) [below=of p_1] {$p_3$};
        \path[->]
        (s')  edge [bend left] node {$a$} (p_1)
              edge node [swap] {$b$} (p_3)
        (p_0) edge node {$a$} (p_1)
              edge node {$b$} (p_3)
        (p_1) edge node [swap] {$b$} (p_2)
              edge node {$a$} (p_3)
        (p_2) edge node {$a$,$b$}  (p_3)
              edge [bend right] node [swap] {$a$} (p_1)
        (p_3) edge [loop below] node {$a$,$b$} ();
    \end{tikzpicture}
    \end{center}

\vspace{15pt}

Разбира се, предложените конструкции строят недетерминирани автомати, а нашата цел
беше да построим детерминирани такива, за да покажем затвореността на класа на
автоматните езици. Както показахме на миналото упражнение, резултатните недетерминирани
автомати винаги могат да се детерминират като последна стъпка в предложените конструкции.

\section{Задачи}
    \textbf{Задача 1.} Дайте \textit{директни} конструкции за затвореността относно
    следните две операции.\\
    (а) \textit{Допълнение.} \\
    (б) \textit{Сечение.} \\

    \vspace{10pt}

    \textit{Забележка.} \textit{Директни} ще рече да не използвате, че сечението може
    да се представи чрез допълнение и обединение.

    \vspace{15pt}

    \textbf{Задача 2.} Използвайки конструкциите от \textbf{Теорема 1}, постройте
    недетерминирани крайни автомати разпознаващи следните езици. \\
    (а) $\{a\}^*\circ(\{ab\}\cup\{ba\}\cup\{\epsilon\})\circ\{b\}^*$ \\
    (б) $(\{a\}\cup\{b\})^*\circ(\{\epsilon\}\cup\{c\})$ \\
    (в) $(\{a\}^*\cup\{b\}^*)\circ\{c\}$ 

    \vspace{20pt}

    \textit{Забележка.} В \textbf{Задача 2} можете неформално да приложите конструкциите върху 
    недетерминирани автомати, вместо да започнете от детерминирани такива и да 
    детерминирате на всяка стъпка. Ако човек е схванал конструкциите за детерминирани
    автомати, не би трябвало да се затрудни да ги прилага и за недетерминирани такива,
    въпреки липсата на формално изложение.

    \vspace{15pt}

    \textbf{Задача 3.} Нека $L \subseteq \Sigma^*$. Дефинираме следните езици. \\
    \textcolor{blue}{\hspace{10pt} 1. $Pref(L) = \{w \in \Sigma^*$ | $x = wy$ за някои $x \in L, y \in \Sigma^*$\}
    (множеството \\ \hspace{22pt} от \textbf{префиксите} на $L$). \\
    \hspace{10pt} 2. $Suff(L) = \{w \in \Sigma^*$ | $x = yw$ за някои $x \in L, y \in \Sigma^*$\}
    (множеството \\ \hspace{22pt} от \textbf{суфиксите} на $L$). \\
    \hspace{10pt} 3. $Subseq(L) = \{w_1w_2...w_k$ | $k \in \mathbb{N}, w_i \in \Sigma^*$ за $i = 1,...,k$,
    и има дума \\ \hspace{22pt} $x = x_0w_1x_1w_2x_2...w_kx_k \in L\}$. (множеството от \textbf{подредиците}
    на $L$). \\
    \hspace{10pt} 4. $Max(L) = \{w \in L$ | ако $x \neq \epsilon$, то $wx \notin L\}$. \\
    \hspace{10pt} 5. $L^{rev} = \{w^{rev}$ | $w \in L\}$.} \\
    Покажете че ако $L$ е автоматен, то всеки от следните езици също е автоматен. \\
    (а) $Pref(L)$ \\
    (б) $Suff(L)$ \\
    (в) $Subseq(L)$ \\
    (г) $Max(L)$ \\
    (д) $L^{rev}$

    \vspace{15pt}

    \textbf{Задача 4.} За два езика $L_1$ и $L_2$ над азбука $\Sigma$, дефинираме 
    \textit{частното} на $L_1$ по отношение $L_2$ да бъде езикът 
    \begin{center}
      $L_1/L_2 = \{w \in \Sigma^*$ | $wx \in L_1$ за някое $x \in L_2$\}.
    \end{center}
    Докажете, че ако $L_1$ е автоматен, то $L_1/L_2$ е автоматен език.


\vspace{25pt}

\section{Решения}
    \textbf{Задача 1.}(а) Достатъчно е да обърнем крайността на всяко състояние. Тоест
    $F' = Q \setminus F$. Тук обаче е ключово да отбележим, че входният автомат трябва
    да бъде детерминиран. \\
    (б) Нека $A_1 = (Q_1,\Sigma,\delta_1,s_1,F_1)$ и $A_2 = (Q_2,\Sigma,\delta_2,s_2,F_2)$.
    Предлагаме НКА $N = (Q_1 \times Q_2,\Sigma,\Delta,(s_1,s_2),F_1 \times F_2)$, където 
    $\Delta$ е дефинирана както следва. \\
    \begin{center}
      $\Delta((q_1,q_2),a) = (\delta_1(q_1,a),\delta_2(q_2,a))$
    \end{center}

    \vspace{100pt}

    \textbf{Задача 2.} 
    (а)
    \begin{center}
      \begin{tikzpicture}[shorten >=1pt,node distance=2cm,on grid,auto] 
        \node[state,initial,accepting] (q_0) {$q_0$}; 
        \node[state] [right=of q_0] (q_1) {$q_1$};
        \node[state] (q_2) [right=of q_1] {$q_2$};
        \node[state,accepting] (q_3) [right=of q_2] {$q_3$};
        \node[state] (q_4) [above=of q_1] {$q_4$};
        \node[state] (q_5) [right=of q_4] {$q_5$};
        \node[state,accepting] (q_6) [right=of q_5] {$q_6$};
        \node[state,accepting] (q_7) [below=of q_1] {$q_7$};
        \node[state,accepting] (q_8) [right=of q_3] {$q_8$};
        \path[->]
        (q_0)  edge [loop above] node {$a$} ()
               edge node {$a$} (q_5)
               edge [bend right,below] node [xshift=0.5cm] {$b$} (q_2)
               edge [bend right=40,below] node {$b$} (q_8)
        (q_1)  edge node {$b$} (q_2)
        (q_2)  edge node {$a$} (q_3)
        (q_3)  edge node {$b$} (q_8)
        (q_4)  edge node {$a$} (q_5)
        (q_5)  edge node {$b$} (q_6)
        (q_6)  edge node {$b$} (q_8)
        (q_7)  edge [bend right,below] node {$b$} (q_8)
        (q_8)  edge [loop above] node  {$b$} ();
    \end{tikzpicture}
    \end{center}

    (б)
    \begin{center}
      \begin{tikzpicture}[shorten >=1pt,node distance=1.7cm,on grid,auto] 
        \node[state,initial,accepting] (q_0) {$q_0$}; 
        \node[state] [right=of q_0] (q_1) {$q_1$};
        \node[state] (q_2) [above right=of q_1] {$q_2$};
        \node[state,accepting] (q_3) [right=of q_2] {$q_3$};
        \node[state] (q_4) [below right=of q_1] {$q_4$};
        \node[state,accepting] (q_5) [right=of q_4] {$q_5$};
        \node[state,accepting] (q_6) [below right=of q_3] {$q_6$};
        \node[state,accepting] (q_7) [above right=of q_6] {$q_7$};
        \node[state] (q_8) [below right=of q_6] {$q_8$};
        \node[state,accepting] (q_9) [right=of q_8] {$q_9$};
        \path[->]
        (q_0)  edge [bend left=50] node {$a$} (q_3)
               edge [bend right=50,below] node {$b$} (q_5)
               edge [bend right=60,below] node {$c$} (q_9)
        (q_1)  edge [bend right=20] node {$a$} (q_3)
               edge [bend left=20] node {$b$} (q_5)
        (q_2)  edge node {$a$} (q_3)
        (q_3)  edge [loop above] node {$a$} ()
               edge [bend right] node {$b$} (q_5)
               edge [bend left=90] node {$c$} (q_9)
        (q_4)  edge node {$b$} (q_5)
        (q_5)  edge [loop below] node {$b$} (q_5)
               edge [bend right] node {$a$} (q_3)
               edge [bend right,below] node {$c$} (q_9)
        (q_6)  edge [bend left] node {$c$} (q_9)
        (q_8)  edge node  {$c$} (q_9);
    \end{tikzpicture}
    \end{center}

    \vspace{1000pt}
    (в)
    \begin{center}
      \begin{tikzpicture}[shorten >=1pt,node distance=2cm,on grid,auto] 
        \node[state,initial] (q_0) {$q_0$}; 
        \node[state] [above right=of q_0] (q_1) {$q_1$};
        \node[state] (q_2) [right=of q_1] {$q_2$};
        \node[state] (q_3) [right=of q_2] {$q_3$};
        \node[state] (q_4) [below right=of q_0] {$q_4$};
        \node[state] (q_5) [right=of q_4] {$q_5$};
        \node[state] (q_6) [right=of q_5] {$q_6$};
        \node[state] (q_7) [above right=of q_6] {$q_7$};
        \node[state,accepting] (q_8) [right=of q_7] {$q_8$};
        \path[->]
        (q_0)  edge [bend right=15] node {$a$} (q_3)
               edge [bend left=15,below] node {$b$} (q_6)
               edge [bend right=90,below] node {$c$} (q_8)
        (q_1)  edge [bend left] node {$a$} (q_3)
               edge [bend left=50] node {$c$} (q_8)
        (q_2)  edge node {$a$} (q_3)
        (q_3)  edge [bend left=20] node {$c$} (q_8)
               edge [bend left=12] node {$a$} (q_2)
        (q_4)  edge [bend right=40] node {$b$} (q_6)
               edge [bend right=50,below] node {$c$} (q_8)
        (q_5)  edge node {$b$} (q_6)
        (q_6)  edge [bend right=20] node {$c$} (q_8)
               edge [bend left=12] node {$b$} (q_5)
        (q_7)  edge node  {$c$} (q_8);
    \end{tikzpicture}
    \end{center}

    \textbf{Задача 3.} (а) Просто променяме множеството от крайните състояния по следния
    начин:
    \begin{center}
      $F' = \{q \in Q$ | $(\exists w \in \Sigma^*)[\hat{\delta}(q,w) \in F]\}$.
    \end{center}
    (б) Едно състояние $q \in Q$ ще наричаме \textit{достижимо от s}, ако съществува дума
    $w \in \Sigma^*$, такава че $\hat{\delta}(s,w) = q$. Нека $\mathbb{Q}$
    е множеството от всички достижими от $s$ състояния на входния автомат 
    (този, разпознаващ $L$). Добавяме ново начално състояние $s \notin Q$. Добавяме $s$ към $F$.
    Ясно е как работи $\Delta$ върху състояния, които не са $s$. Нека $a \in \Sigma$. Имаме
    \begin{center}
      $\Delta(s,a) = \{\delta(q,a)$ | $q \in \mathbb{Q}\}$.
    \end{center}
    (в) Нека $Q = \{q_1,...,q_n\}$. За $i=1,...,n$ нека $Q_i$ е множеството от достижимите
    от $q_i$ състояния. Нека $a \in \Sigma$. Тогава
    \begin{center}
      $\Delta(q_i,a) = \bigcup\limits_{q \in Q_i} \{\delta(q,a)\}$ 
    \end{center}
    (г) Променяме множеството от крайните състояния по следния начин:
    \begin{center}
      $F' = \{q \in F$ | $(\forall w \in \Sigma^+)[\hat{\delta}(q,w) \notin F]\}$.
    \end{center}
    (д) Първо обръщаме преходите на автомата:
    \begin{center}
      $\Delta(q,a) = \{p \in Q$ | $\delta(p,a) = q\}$.
    \end{center}
    След това добавяме ново начално състояние $s \notin Q$, което е крайно тстк началното
    състояние на входния автомат е крайно. Към $\Delta$ добавяме преходите
    \begin{center}
      $\Delta(s,a) = \bigcup\limits_{q \in F} \Delta(q,a)$,
    \end{center}
    за всяко $a \in \Sigma$.

\end{document} 