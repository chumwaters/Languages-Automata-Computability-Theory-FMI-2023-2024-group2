\documentclass{article}
\usepackage[document]{ragged2e}
\usepackage[utf8]{inputenc}
\usepackage[russian]{babel}
\usepackage{amssymb}
\usepackage{xcolor}
\usepackage{mathtools}
\usepackage{tikz}
\usepackage{qtree}
\usepackage{amsmath}
\usepackage{mathrsfs}

\usetikzlibrary{automata,positioning}

\newcommand{\cleft}[2][.]{%
  \begingroup\colorlet{savedleftcolor}{.}%
  \color{#1}\left#2\color{savedleftcolor}%
}
\newcommand{\cright}[2][.]{%
  \color{#1}\right#2\endgroup
}

\newcommand{\bleft}{
    \boldsymbol{\left(\right.}
}

\newcommand{\bright}{
    \boldsymbol{\left.\right)}
}

\newcommand{\bplus}{
    \boldsymbol{+}
}

\newcommand{\bepsilon}{
    \boldsymbol{\epsilon}
}

\begin{document}
\begin{center}
    {\huge Упражнение 12}
\end{center}

\vspace{15pt}

\section{Синтактични дървета на извод}
\hspace{15pt} Нека $G$ е контекстно-свободна граматика. Както вече видяхме, една дума
$w \in L(G)$ може да има няколко извода по $G$. Да разгледаме по-прост пример отново 
използващ граматиката, генерираща думите от балансирани леви и десни скоби — думата
$()()$ има два различни извода, именно, \\

\begin{center}
    $S \Rightarrow SS \Rightarrow (S)S \Rightarrow ()S \Rightarrow ()(S) \Rightarrow ()()$
\end{center}

и

\begin{center}
    $S \Rightarrow SS \Rightarrow S(S) \Rightarrow (S)(S) \Rightarrow (S)S \Rightarrow ()()$
\end{center}
    
В действителсност, тези два извода са в някакъв смисъл "еднакви". Използваните правила
са едни и същи и се прилагат на едни и същи позиции в междинните низове. Единствената
разлика е в \textit{реда}, в който тези правила се прилагат. Интуитивно, и двата 
извода могат да се получат чрез обхождане на следното дърво.
\vspace{25pt}

\Tree [.$S$ [.$S$ ( [.$S$ $\epsilon$ ] ) ] [.$S$ ( [.$S$ $\epsilon$ ] ) ] ]

\vspace{15pt}

\hspace{15pt} Неформално, една такава картинка наричаме \textbf{синтактично дърво 
на извод}. Точките наричаме \textbf{върхове}; всеки връх има етикет, който е елемент
на $\Sigma \cup V$. Най-горния връх наричаме \textbf{корен}, а най-долните върхове 
наричаме \textbf{листа}. Всички листа са етикетирани с терминали или с $\epsilon$.
Конкатенирайки етикетите на листата в ред от ляво надясно, получаваме изведената 
дума, която още ще наричаме \textbf{продукт} на дървото.
\hspace{15pt} По-формално, за дадена контекстно-свободна граматика $G = (V,\Sigma,R,S)$,
дефинираме синтактично дърво на извод заедно с неговите корен, листа и продукт 
индуктивно, както следва. \\

\vspace{10pt}

1.
\Tree [.$\circ\;\sigma$ ]

\vspace{5pt}

Това е синтактично дърво за всяко $\sigma \in \Sigma$. Единственият връх на това
дърво е едновременно негов корен и единствено листо. Продукта на дървото е $\sigma$. \\

\vspace{10pt}

2. Ако $A \rightarrow \epsilon$ е правило в $R$, то \\

\vspace{5pt}

\Tree[.$A$ $\epsilon$ ] \\

\vspace{5pt}

е синтактично дърво на извод; корена му е върхът с етикет $A$, единственото му листо
е върхът с етикет $\epsilon$, и продукта му е $\epsilon$. \\

\vspace{10pt} 

3. Ако \\
\vspace{5pt}
\begin{center}
\qroof{$y_1$}.$A_1$ \hspace{15pt} \qroof{$y_2$}.$A_2$ \hspace{15pt}...\hspace{15pt}\qroof{$y_n$}.$A_n$
\end{center}
\vspace{5pt}

са синтактични дървета, за $n \geq 1$, с корени етикетирани $A_1,...,A_n$ съответно, и
продукти $y_1,...,y_n$, и $A \rightarrow A_1...A_n$ е правило в $R$, то \\

\vspace{5pt}
\Tree[.$A$ [\qroof{$y_1$}.$A_1$ ] [\qroof{$y_2$}.$A_2$ ] [\qroof{$y_n$}.$...\;A_n$ ] ]
\vspace{5pt}

е синтактично дърво на извод. Корена му е новият връх етикетиран $A$, листата му са 
листата на съставящите го синтактични дървета, и продукта му е $y_1...y_n$.
\vspace{25pt}

\section{Езици, които не са контекстно-свободни} 
\hspace{15pt} Множеството на контекстно-свободните езици над дадено множество от 
терминали и нетерминали $\Sigma \cup V$ също е изборимо безкрайно.
За жалост нямаме толкова подходящо описание на контекстно-свободните езици, каквито 
бяха регулярните изрази за регулярните езици, с помощта на което да съобразим този факт. Достатъчно е
обаче да забелижим, че множеството на всички възможни правила над $\Sigma \cup V$ е 
$V \times (\Sigma \cup V)^*$, което е изброимо безкрайно, по следните съображения: \\
— $V$ е крайно; \\
— $\Sigma$ е крайно; \\
— $\Sigma \cup V$ е крайно м-во, като обединение на крайни м-ва; \\
— $(\Sigma \cup V)^*$ е изброимо безкрайно; \\
— декартовото произведение на крайно множество с избоимо безкрайно такова е изброимо безкрайно. \\
\hspace{15pt} Правилата на една граматика са \textit{крайно} подмножество на изброимо безкрайното множество 
$V \times (\Sigma \cup V)^*$. Тоест, множеството от всички възможни множества от правила
е множеството от всички \textit{крайни} подмножества на изброимо безкрайно множество. Като такова, 
то е изброимо безкрайно (всяко крайно подмножество може да се кодира в разлагане на 
прости множители на някое естествено число; това кодиране е съобразено с даденото ни
изброяване). Възможните множества от правила образуват изброимо безкрайно множество. 
Остана само да съобразим, че възможните избори на начална променлива образуват крайно
множество и можем да заключим, че има изброимо много контекстно-свободни граматики
при фиксирано множество от терминали и нетерминали, а от тук и контекстно-свободните
езици са изброимо много.\\
\hspace{15pt} Щом множеството на контекстно-свободните езици е изброимо безкрайно, а 
множеството на всички езици е неизброимо безкрайно, то съществуват езици, които не са 
контекстно-свободни. За да открием критерий за това, кога един език не е 
контекстно-свободен отново е смислено да търсим свойство, което всички безкрайни
контекстно-свободни езици притежават. \\
\hspace{15pt} Нека $G = (V,\Sigma,R,S)$ е контекстно-свободна граматика. С $\phi(G)$ означаваме 
най-големият брой символи в дясната страна на кое да е правило в $R$. \textbf{Път} 
в синтактично дърво на извод е редица от върхове свързани със линия в дървото на 
извод; първия връх е коренът, а посления е листо. \textbf{Дължината} на този път е
броят на свързващите линии в него. \textbf{Височината} на едно синтактично дърво е
дължината на най-дългия път в него. \\

\vspace{15pt}

\textbf{Лема 1.} Продуктът на дадено синтактично дърво на $G$ с височина $h$ има дължина
не повече от $\phi(G)^h$.

\vspace{15pt}
\hspace{15pt} Доказателството на горната лема е проста индукция по $h \geq 1$. Лемата 
ни казва, че синтактичното дърво на коя да е дума $w \in L(G)$ с $|w| > \phi(G)^h$ трябва да
има път по-дълъг от $h$. Това е ключово за доказателството на следната теорема, 
представляваща търсения от нас критерии за контекстно-свободност. \\

\vspace{15pt}

\textbf{Теорема 1(лема за разрастването за контекстно-свободни езици).} Нека 
$L$ е контекстно-свободен език. Тогава съществува естествено 
число $p \geq 1$, такова че всяка дума $w \in L$ с $|w| \geq p$ може да се презапише
във вида $w = vwxyz$, като \\
(1) $|wy| \geq 1$, \\
(2) $|wxy| \leq p$ и \\
(3) $(\forall i \in \mathbb{N})[vw^ixy^iz \in L]$. \\

\vspace{15pt}

\hspace{15pt}Нека $G = (V,\Sigma,R,S)$ е такава контекстно-свободна граматика, че $L(G) = L$. 
Ясно е, че един възможен избор за $p$ е именно $\phi(G)^{|V|}+1$ — ако $w$ е дума от 
$L(G)$, такава че $|w| \geq \phi(G)^{|V|}+1$, то всяко синтактично дърво за $w$ ще има
продукт с дължина поне $\phi(G)^{|V|}+1$, тоест по-голяма от $\phi(G)^{|V|}$. Съгласно
\textbf{Лема 1}, всяко такова дърво трябва да има път по-дълъг от $|V|$, тоест с дължина
поне $|V| + 1$. Този път ще съдържа $|V| + 2$ върха, от които точно един — листото — е 
етикетиран с терминал, а останалите $|V| + 1$ са етикетирани с нетерминали. Тогава е
ясно, че поне два върха по пътя ще са етикетирани с един и същ нетерминал. Ако 
$v_1$ и $v_2$ са два такива върха, то заменяйки "поддървото, вкоренено във $v_1$" \hspace{0,01cm} с
"поддървото, вкоренено във $v_2$" \hspace{0,01cm} получавамe синтактично дърво на извод за $vw^0xy^0z$.
Заменяйки $i$ на брой пъти последователно "поддървото, вкоренено във $v_2$" \hspace{0,01cm} с
"поддървото, вкоренено във $v_1$" \hspace{0,01cm} получаваме синтактично дърво на извод
за $vw^{i+1}xy^{i+1}z$.

\vspace{15pt}

\textbf{Пример 1.} $L = \{a^nb^nc^n$ | $n \in \mathbb{N}\}$ не е контекстно-свободен. За да докажем
това, да допуснем, че $L$ контекстно-свободен и нека $p$ е въпросното число от 
\textbf{лемата за разрастването}. Съгласно лемата, думата $w = a^pb^pc^p$ може да се 
презапише във вида $w = vwxyz$ за $wy \neq \epsilon$ и $vw^ixy^iz \in L$ за всяко
$i \in \mathbb{N}$. Има два случая, всеки от които води до противоречие. \\
\textbf{1сл.} $wy$ съдържа появи на всяка от трите букви $a,b$ и $c$. Тогава поне една
измежду $w$ и $y$ трябва да съдържа появи на поне две от тези букви. Веднага се вижда, че 
редът на буквите в $vw^2xy^2z$ е развален — има $b$ преди $a$, или $c$ преди $a$ или $b$. \\
\textbf{2сл.} $wy$ съдържа появи само на някои, но не всяка от буквите $a,b$ и $c$. 
Тогава веднага можем да съобразим, че $vw^2xy^2z$ ще има неравен брой $a$-та, $b$-та и
$c$-та.


\vspace{15pt}

\textbf{Пример 2.} $L = \{a^n$ | $n \geq 1$ е просто число\} не е контекстно-свободен.
За да покажем това, да допуснем, че $L$ е контекстно-свободен и нека $p$ е въпросното
число от \textbf{лемата за разрастването}. Нека $p'$ е първото просто число по-голямо
от $p$. Тогава думата $a^{p'}$ може да се презапише във вида $w = vwxyz$, където $wy \neq \epsilon$
. Тогава $wy = a^q$ и $vxz = a^r$, където $q$ и $r$ са естествени числа и $q > 0$. Съгласно
лемата, $vw^ixy^iz \in L$ за всяко $i \in \mathbb{N}$. Тоест, $r + iq$ е просто число
за всяко $i \in \mathbb{N}$. За $i = q+r+1$ обаче имаме, че \\
\begin{center}
$r + iq = $ \\
$r + (q+r+1)q = $ \\
$r + q^2 + rq + q = $ \\
$r(q+1) + q(q+1) = $ \\
$(q+1)(r+q)$, \\
\end{center}
което е произведение на две числа, всяко от които е по-голямо от нула и значи няма как
да е просто число. Противоречие, породено от допускането ни, че $L$ е контекстно-свободен 
език.
\vspace{25pt}

\section{Задачи} 
\textbf{Задача 1.} Използвайте лемата за разрастването за да докажете, че следните езици
не са контекстно-свободни. \\
(а) $\{a^{n^2}$ | $n \in \mathbb{N}\}$ \\
(б) $\{www$ | $w \in \{a,b\}^*\}$ \\
(в) $\{w \in \{a,b,c\}^*$ | $w$ съдържа равен брой $a$-та, $b$-та и $c$-та \} \\

\vspace{15pt}

\textbf{Задача 2.} Използвайте лемата за разрастването за да докажете, че езикът 
$L = \{babaabaaab...ba^{n-1}ba^nb$ | $n \geq 1$\} не е контекстно-свободен.

\vspace{15pt}

\textbf{Задача 3.} Кои от следните езици са контекстно-свободни? Обосновете се. \\
(а) $\{a^mb^nc^p$ | $m = n$ или $n = p$ или $m = p$\} \\
(б) $\{a^mb^nc^p$ | $m \neq n$ или $n \neq p$ или $m \neq p$\} \\
(в) $\{a^mb^nc^p$ | $m = n \; \& \; n = p \; \& \; m = p$\} \\
\vspace{25pt}



\section{Решения}
    \textbf{Задача 1.} (а) Да допуснем, че $L$ е контекстно-свободен. 
    Разглеждаме думата $a^{p^2} \in L$. Съгласно лемата за 
    разрастването $a^{p^2} = vwxyz$ за някои $v,w,x,y,z \in \{a\}^*$. Нека
    $w = a^k$ и $y = a^l$ съгласно свойства (1) и (2) от лемата, $1 \leq k + l \leq p$.
    От тук имаме, че $p^2 + 1 \leq p^2 + k + l \leq p^2 + p$. Тоест, 
    $p^2 < p^2 + k + l \leq p^2 + p < p^2 + 2p + 1$. Значи 
    $p^2 < p^2 + k + l < (p+1)^2$. Тоест $p^2 < |vw^2xy^2z| < (p+1)^2$, откъдето
    следва, че $|vw^2xy^2z|$ няма как да бъде точен квадрат. Следователно
    $vw^2xy^2z \notin L$. Намерихме число, за което прилагайки свойство (3) стигаме
    до дума извън $L$. Противоречие с лемата. Значи $L$ не е контекстно-свободен. \\
    \vspace{5pt}
    (б) Разглеждаме думата $(a^pb)^3 \in L$. За $wxy$ имаме следните два случая: \\
    (1) $wxy = a^k$, аз някое $1 \leq k \leq p$. Тогава лесно може да се 
    съобрази, че покачвайки нагоре ще излезем от езика. \\
    (2) $wxy = a^kba^l$ за някои $1 \leq k + l + 1 \leq p$. Имаме следните 
    подслучаи: \\
    \hspace{15pt}(2.1) Буквата $b$ е в $w$ или в $y$. Тогава думата $vw^0xy^0z$ ще има
    две $b$-та и няма как да е в $L$. \\
    \hspace{15pt}(2.2) Буквата $b$ е в $x$. Тогава както и да изберем да се възползваме
    от свойство (3) на лемата, ще стигнем до дума извън $L$, защото броят на $a$-тата
    няма да е един и същ преди всяко от трите $b$-та, а е ясно, че всяка разбивка на
    покачената дума на три еднакви части $www$ ще има свойството, че $w$ завършва на 
    $b$.\\
    \vspace{5pt}
    (в) Разглеждаме думата $a^pb^pc^p \in L$. За $wxy$ имаме следните два случая: \\
    (1) $wxy = \sigma^k$ за някои $\sigma \in \{a,b,c\}$ и $1 \leq k \leq p$. Тук е 
    ясно, че както и да покачим, излизаме от езика. \\
    (2) $wxy$ попада в интервал, застъпващ \textit{точно} две различни букви. 
    Независимо как са разпределени тези две букви между $w$ и $y$, тъй като
    $|wy| \geq 1$, то покачвайки в коя да е посока ще получим дума, в която бройките
    на трите различни букви не са равни.

    \vspace{15pt}

    \textbf{Задача 2.} Да разгледаме думата $babaabaaab...ba^{p-1}ba^pb \in L$. Нека
    $vwxyz$ е нейно разбиване със свойствата от лемата за разрастването. Случаите, в
    които $wxy$ попада изцяло или частично преди $a^pb$ са ясни — покачвайки в която
    и да е посока, линейно нарастващата структура на думата се нарушава и попадаме 
    извън езика. Да разгледаме случая, в който $wxy$ попада изцяло в $a^pb$. Имаме
    следните подслучаи: \\
    (1) $wxy = a^k$ за някое $1 \leq k \leq p$. Тук нещата пак са ясни — ако се опитаме
    да приложим св-во (3) веднага разваляме линейно нарастващата структура на $a$-тата. \\
    (2) $wxy = a^kb$ за някое $1 \leq k \leq p$. Ако $w$ и $y$ са едновременно 
    непразни, то в частност $y$ съдържа $b$-то, а $w = a^l$, за някое $l \leq k$ и покачвайки нагоре добавяме $a$-та преди
    последното $b$ в оригиналната дума, което разваля структурата. Ако само $y$ е празна,
    то $x$ съдържа $b$-то и отново имаме същия проблем. Ако и $x$ и $y$ са празни,
    то $w = a^kb$ и покачвайки нагоре получаваме думата $babaabaaab...ba^{p-1}ba^pba^kb$,
    която не е в $L$, тъй като $k \leq p$, тоест $k \neq p+1$. Ако $w$ е празна, а 
    $y$ е непразна, то $y = a^lb$, за някое $l \leq k$ и покачвайки нагоре получаваме
    същия проблем. С това случаите се изчерпаха.

    \vspace{15pt}

    \textbf{Задача 3.} (а) Този език е $\{a^mb^mc^n$ | $m,n \in \mathbb{N}\} \cup \{a^mb^nc^n$ | $m,n \in \mathbb{N}\} \cup \{a^mb^nc^m$ | $m,n \in \mathbb{N}\}$
    и следователно е контекстно-свободен. Граматики ще дадем за първия и третия от тези операнди. \\
    (1) $S \rightarrow Sc \; | \; A$ \\
    \hspace{15pt}$A \rightarrow aAb \; | \; \epsilon$. \\
    (2) $S \rightarrow aSc \; | \; B$ \\
    \hspace{15pt}$B \rightarrow bB \; | \epsilon$. \\
    \vspace{5pt}
    (б) Този език е $\{a^mb^nc^p$ | $m \neq n\} \cup \{a^mb^nc^p$ | $n \neq p\} \cup \{a^mb^nc^p$ | $m \neq p\}$.
    На свой ред той е равен на $[\{a^mb^n \; | \; m \neq n\} \circ \mathscr{L}(c^*)] \cup [\mathscr{L}(a^*) \circ \{b^nc^p \; | \; n \neq p\}] \cup \{a^mb^nc^p$ | $m \neq p\}$ 
    . Всеки от тези езици знаем вече, че е контекстно-свободен, освен $\{a^mb^nc^p \; | \; m \neq p\}$. 
    Контекстно-свободна граматика за този език е например: \\
    \begin{center}
    $S \rightarrow aA \; | \; Cc$ \\
    $A \rightarrow aA \; | \; aAc \; | \; B$ \\
    $C \rightarrow Cc \; | \; aCc \; | \; B$ \\
    $B \rightarrow bB | \epsilon$ \\
    \end{center}
    Следователно и целият език е контекстно-свободен. \\ 
    \vspace{5pt}
    (в) Това е езикът $\{a^nb^nc^n \; | \; n \in \mathbb{N}\}$, за който вече показахме,
    че не е контекстно свободен. \\
    \vspace{5pt}
\end{document} 