\documentclass{article}
\usepackage[document]{ragged2e}
\usepackage[utf8]{inputenc}
\usepackage[russian]{babel}
\usepackage{amssymb}
\usepackage{xcolor}
\usepackage{mathtools}
\usepackage{tikz}
\usepackage{amsmath}
\usepackage{mathrsfs}
\usetikzlibrary{automata,positioning}

\newcommand{\cleft}[2][.]{%
  \begingroup\colorlet{savedleftcolor}{.}%
  \color{#1}\left#2\color{savedleftcolor}%
}
\newcommand{\cright}[2][.]{%
  \color{#1}\right#2\endgroup
}

\newcommand{\bleft}{
    \boldsymbol{\left(\right.}
}

\newcommand{\bright}{
    \boldsymbol{\left.\right)}
}

\newcommand{\bplus}{
    \boldsymbol{+}
}

\newcommand{\bepsilon}{
    \boldsymbol{\epsilon}
}

\begin{document}
\begin{center}
    {\huge Упражнение 7}
\end{center}

\vspace{15pt}

\section{Критерии за нерегулярност}
    \hspace{15pt} Вече разгледахме два метода, чрез които показваме регулярността на 
    даден език. Регулярните езици над дадена азука $\Sigma$ обаче са изброимо много (защото 
    регулярните изрази над $\Sigma$ са изброимо много). От друга страна, 
    множеството на всички езици над $\Sigma$, $\mathscr{P}(\Sigma^*)$, е неизброимо
    безкрайно като степенно множество на изброимо безкрайно множество, $\Sigma^*$.
    Тези съображения показват съществуването на нерегулярни езици над $\Sigma$.
    Ясно е, че каквато и да е $\Sigma$ и каквито и да са тези езици, те най-малкото 
    ще бъдат безкрайни. Но какво откроява безкрайните нерегулярни езици от безкрайните
    регулярни такива? Както знаем, всеки регулярен език се разпознава от ДКА. Ако $L$
    е безкраен регулярен език, а $A$ е ДКА с $n$ състояния, такъв че $L(A) = L$, 
    то безкрайността на $L$ влече съществуването на думи с дължина поне $n$. Нека
    $w$ е една такава дума. Тогава пътят на $w$ по $A$ със сигурност минава през поне
    $n+1$ състояния. По принципа на Дирихле, някое от състоянията по този път ще се 
    повтаря; тоест пътят на $w$ по $A$ ще съдържа цикъл. Естествено движейки се по $A$
    ние можем да изберем да се "завъртим"\hspace{0.01cm} по този цикъл произволен брой пъти, след което
    наличието на $w$ в $L$ ни гарантира, че ще можем да приключим този път в крайно
    състояние, ефективно вкарвайки безкрайно много думи (за всяко количество "завъртания"),
    зависими по някакъв начин от $w$, в $L$. Всичко това можем да изкажем чрез следната
    теорема.

    \vspace{15pt}

    \textbf{Теорема 1(лема за разрастването/the pumping lemma).} Нека $L$ е регулярен
    език. Тогава съществува естествено число $p \geq 1$, такова че всяка дума $w \in L$
    с $|w| \geq p$ може да се презапише във вида $w = xyz$, като \\
    (1) $|y| \geq 1$, тоест $y \neq \epsilon$; \\
    (2) $|xy| \leq p$; \\
    (3) $xy^iz \in L$ за всяко $i \in \mathbb{N}$.

    \vspace{15pt}

    \hspace{15pt}Първия (най-малкия) свидетел за съществуването на числото $p$ е \\
    именно броят на състоянията на минималния автомат за $L$. Такъв автомат знаем че 
    съществува, заради регулярността на $L$. Мислейки за минималния автомат $A$ за $L$,
    всяка дума $w \in L$ с $|w| \geq p$ ще съдържа цикъл в пътя си по $A$. Представянето
    $w = xyz$ заедно със свойствата си (1), (2) и (3) точно съответства на този факт.
    Наистина, $x$ е думата прочетена от началното състояние до началото на \textit{първия}
    цикъл в пътя на $w$ по $A$; 
    $y$ е думата прочетена по този цикъл (затова $|y| \geq 0$), а $z$ е думата прочетена
    от края на цикъла до крайното състочние, в което $w$ приключва пътя си по $A$. 
    Свойството (2) следва от избора ни $y$ да е думата прочетена по първия цикъл в пътя
    на $w$ по $A$ — първото повторение на състояние ще се случи след прочитането на $y$,
    тоест $|xy|$ няма как да бъде по-голяма от броя на състоянията на $A$. Свойството
    (3) е именно свойството, което дава и името на теоремата; щом можем да се въртим
    по така фиксирания първи цикъл в пътя на $w$ по $A$ произволен брой пъти, четейки
    по него думата $y$, то за всяко $i \in \mathbb{N}$, думата $xy^iz$ ще да бъде в
    $L(A)$, а значи и в $L$. \\

    \vspace{15pt}

    \hspace{15pt} \textbf{Теорема 1} по същество е импликация от вида 
    \begin{center}
        $L$ е регулярен $\implies P(L)$. 
    \end{center}
    Разбира се в сила е и нейната контрапозиция
    \begin{center}
        $\neg P(L) \implies$ $L$ не е регулярен.
    \end{center}
    \textit{Това} е първият критерии за нерегулярност, който ще разгледаме. Отрицанието
    на свойството $P$ трябва да се разгледа по-подробно. Първо, да запишем свойството
    $P$ формално.
    \begin{center} 
        $P(L) \iff (\exists p \in \mathbb{N}^+)(\forall w \in L)[|w| \geq p \implies (\exists x \in \Sigma^*)(\exists y \in \Sigma^*)(\exists z \in \Sigma^*)[w = xyz$ $\&$ $|y| \geq 1$ $\&$ $|xy| \leq p$ $\&$ $(\forall i \in \mathbb{N})[xy^iz \in L]]]$
    \end{center}
    Сега можем да съобразим, че отрицанието на $P$ е точно
    \begin{center}
        $\neg P(L) \iff (\forall p \in \mathbb{N}^+)(\exists w \in L) [|w| \geq p$ $\&$ $(\forall x \in \Sigma^*)(\forall y \in \Sigma^*)(\forall z \in \Sigma^*)[w = xyz$ $\&$ $|y| \geq 0$ $\&$ $|xy| \leq p$ $\implies$ $(\exists i \in \mathbb{N})[xy^iz \notin L]]]$.
    \end{center}

    Ако успеем да покажем, че даден език притежава горното свойство, то контрапозицията
    на лемата за разрастването ни дава, че този език не \\ е регулярен. \\
    \hspace{15pt} Първо трябва да 
    удовлетворим кванторите в началото на формулата \\ $\boxed{(\forall p \in \mathbb{N}^+)}$ и
    $\boxed{(\exists w \in L)}$, както и първия конюнкт в областта им на действие 
    $\boxed{|w| \geq p}$. За целта избираме в ролята на $w$ подходяща дума с дължина по-голяма от и зависеща от $p$
    в $L$. Зависимостта на дължината на $w$ от $p$ е това, което удовлетворява първия
    квантор. \\ 
    \hspace{15pt} След това удовлетворяваме втория конюнкт в областта на действие на  
    $\boxed{(\exists w \in L)}$. За целта разглеждаме произволно представяне на 
    така избраната $w$
    във вида $w = xyz$ със свойствата (1) и (2) и показваме такова естествено число $i$,
    че $xy^iz \notin L$.

    \vspace{15pt}

    \textbf{Пример 1.} Ще докажем посредством \textbf{Теорема 1}, че езикът \\
    $L = \{a^nb^n$ | $n \in \mathbb{N}\}$ не е регулярен, като покажем, че $L$ не
    притежава свойството $P$ (тоест, че $L$ притежава свойството $\neg P$). Нека $p \in \mathbb{N^+}$. Да разгледаме думата 
    $w = a^pb^p \in L$. Очевидно $|w| \geq p$. Сега да разгледаме някое представяне
    $w = xyz$ на $w$ със свойствата (1) и (2). Щом $|xy| \leq p$ и $|y| \geq 1$, то 
    $y = a^k$ за някое $k$, такова че $1 \leq k \leq p$. Тогава за $i = 2$ имаме, че 
    $xy^2z = a^{p+k}b^p \notin L$. 

    \vspace{15pt}

    \hspace{15pt} Сега да си припомним, че \textbf{теоремата на Майхил-Нероуд} 
    \textit{също} беше импликация от вида 
    \begin{center}
        $L$ е регулярен $\implies N(L)$,
    \end{center}
    където $N(L)$ е следното свойство:
    \begin{center}
        $N(L) \iff$ съществува ДКА за $L$ с толкова на брой състояния, колкото са 
        класовете на еквивалентност на $\approx_L$.
    \end{center}
    Това свойство влече, че броят на класовете на еквивалентност на $\approx_L$ \\
    трябва да бъде краен. Тоест имаме импликацията
    \begin{center}
        $N(L) \implies \approx_L$ има краен брой класове на еквивалентност. 
    \end{center}
    Двете импликации дотук ни дават общо
    \begin{center}
        $L$ е регулярен $\implies \approx_L$ има краен брой класове на еквивалентност.
    \end{center}
    Значи в сила е и контрапозитивното
    \begin{center}
        $\approx_L$ има безкраен брой класове на еквивалентност $\implies L$ не е регулярен.
    \end{center}
    \textit{Това} е вторият критерии за нерегулярност, който ще разгледаме.
    Но как можем да покажем, че даден език има безкраен брой класове на еквивалентност
    по релацията си на Нероуд? Идеята е да намерим безкрайна редица от думи 
    $w_1,w_2,...$, такава че всеки две думи в нея са в различен клас на еквивалентност
    по $\approx_L$. Най-простият начин е да създадем подходяща зависимост между дължината
    на думата $w_i$ и индекса ѝ — числото $i$.

    \vspace{15pt}

    \textbf{Пример 2.} Ще покажем отново, че езикът $L = \{a^nb^n$ | $n \in \mathbb{N}$\}
    не е регулярен; този път използвайки новия критерии.
    Да разгледаме редицата $w_1,w_2,...$, за която 
    \begin{center}
        $w_i = a^i$, за всяко $i \in \mathbb{N}$.
    \end{center}
    Нека $i,j \in \mathbb{N}$ са такива, че $i \neq j$. Ще покажем, че $[w_i]_L \neq [w_j]_L$.
    За целта е достатъчно да покажем, че $w_i \notin [w_j]_L$, тъй като $w_i \in [w_i]_L$.
    Еквивалентно на това, трябва да покажем, че $w_i \not \approx_L w_j$. Да разгледаме
    думата $b^i \in \Sigma^*$. От една страна $a^i\textcolor{blue}{b^i} \in L$. От друга страна $a^j\textcolor{blue}{b^i} \notin L$,
    тъй като $i \neq j$. Значи действително $w_i \not \approx_L w_j$. Така показахме,
    че $\approx_L$ има безкраен брой класове на еквивалентност. Следователно $L$ не е
    регулярен.
    
    \vspace{15pt}

    \textbf{Пример 3.} Ще покажем и трети начин, по който можем да показваме, че даден
    език е нерегулярен. Този начин е да се възползваме от регулярните операции и езици,
    за които вече сме показали, че са нерегулярни. Например да разгледаме езика
    \begin{center}
        $L = \{w \in \{a,b\}^*$ | $w$ има равен брой $a$-та и $b$-та\}.
    \end{center}
    $L$ не е регулярен, защото ако беше, то такъв би бил и $L \cap a^*b^*$—по затвореност
    на регулярните езици относно сечение. Обаче, $L \cap a^*b^* =$ \\
    $\{a^nb^n$ | $n \in \mathbb{N}$\},
    за който вече два пъти показахме, че не е регулярен.
\vspace{25pt}

\section{Задачи}
    \textbf{Задача 1.} Използвайте лемата за разрастването или теоремата на \\Майхил-Нероуд,
    за да покажете, че следните езици не са регулярни. \\
    (а) \{$ww$ | $w \in \{a,b\}^*$\} \\
    (б) \{$ww^{rev}$ | $w \in \{a,b\}^*$\} \\
    (в) \{$w\overline{w}$ | $w \in \{a,b\}^*$\}, където $\overline{w}$ е думата получена от $w$, 
    като сме заменили всяко срещане на $a$ с $b$ и обратно \\
    (г) \{$w^{|w|}$ | $w \in \{a,b\}^*$\}.

    \vspace{15pt}

    \textbf{Задача 2.} Докажете, че езикът $\{a^nba^mba^{n+m}$ | $n,m \geq 1\}$ не е 
    регулярен.

    \vspace{15pt}

    \textbf{Задача 3.} Една дума $w$ над $\{a,b\}$ наричаме \textit{балансирана}, ако следните
    две неща са изпълненини: \\
    (1) във всеки префикс на $w$, броят на $a$-тата е не по-малък от броя на $b$-тата; \\
    (2) броят на $a$-тата в $w$ е равен на броя на $b$-тата. \\
    Докажете, че езикът \{$w \in \{a,b\}^*$ | $w$ е балансирана\} не е регулярен.

    \vspace{15pt}

    \textbf{Задача 4.} Кои от следните твърдения са верни? Аргументирайте се.\\
    (а) Всяко подмножество на регулярен език е регулярен език. \\
    (б) Всеки регулярен език си има собствено подмножество, което е регулярен език. \\
    (в) Ако $L$ е регулярен, то регулярен е и \{$xy$ | $x \in L$ $\&$ $y \notin L$\}. \\
    (г) \{$w$ | $w = w^{rev}$\} е регулярен. \\
    (д) Ако $L$ е регулярен език, то такъв е и \{$w$ | $w \in L$ $\&$ $w^{rev} \in L$\}. \\
    (е) Ако $R$ е множество от регулярни езици, то $\bigcup R$ е регулярен език. \\
    (ж) \{$xyx^{rev}$ | $x,y \in \Sigma^*$\} е регулярен.
\vspace{25pt}

\section{Решения}
    \textbf{Задача 1.} Ако използвате PL едни възможни избори са: \\
    (a) $w = a^pba^pb$, $i = 2$; \\
    (б) $w = a^pbba^p$, $i = 2$; \\
    (в) $w = a^pb^p$, $i = 2$; \\
    (г) $w = (a^pb)^{p+1}, i = 2$. \\
    Ако използвате Майхил-Нероуд: \\
    (а) $w_i = a^ib$, за $i \in \mathbb{N}$; конкатенирате $a^ib$ към $w_i$ и $w_j$ за $i \neq j$; \\
    (б) $w_i = a^ib$, за $i \in \mathbb{N}$; конкатенирате $ba^i$ към $w_i$ и $w_j$ за $i \neq j$; \\
    (в) $w_i = a^i$, за $i \in \mathbb{N}$; конкатенирате $b^i$ към $w_i$ и $w_j$ за $i \neq j$; \\
    (г) $w_i = (a^ib)^{i}$, за $i \in \mathbb{N}$; конкатенирате $a^ib$ към $w_i$ и $w_j$ за $i \neq j$; \\
    \vspace{5pt}
    Алтернативен подход за (в) е да пресечем езика с $a^*b^*$, получавайки \\ $\{a^nb^n$ | $n \in \mathbb{N}$\}.

    \vspace{15pt}

    \textbf{Задача 2.} С PL: $w = a^pba^pba^{2p}$, $i = 2$. С Майхил-Нероуд: $w_i = a^{i+1}ba^{i+1}b$,
    за $i \in \mathbb{N}$ и конкатенирате $a^{2(i+1)}$ към $w_i$ и $w_j$ за $i \neq j$.

    \vspace{15pt}

    \textbf{Задача 3.} Пресичаме езика с $a^*b^*$ и получаваме $\{a^nb^n$ | $n \in \mathbb{N}$\}.

    \vspace{15pt}

    \textbf{Задача 4.} (а) Не е вярно. Най-просто, $\Sigma^*$ е регулярен език и всеки
    език над $\Sigma$ е подмножество на $\Sigma^*$. Но както показахме в това упражнение,
    над $\Sigma$ съществуват и нерегулярни езици. \\
    (б) Не е вярно. Тривиално, $\varnothing$ е регулярен език, който няма собствени
    подмножества. \\
    (в) Вярно е. Имаме \\
    \begin{center}
        \{$xy$ | $x \in L$ $\&$ $y \notin L\} =$ \\
        \{$xy$ | $x \in L$ $\&$ $y \in \overline{L}\} =$ \\
        \{$x$ | $x \in L$\} $\circ$ \{$y$ | $y \in \overline{L}\} =$ 
        $L\overline{L}$. 
    \end{center}
    Но ние вече показахме, че регулярните езици са затворени относно операциите
    допълнение и конкатенация. \\
    (г) Не е вярно. Ако пресечем този език с езика на думите с четна дължина, $(aa + ab + ba + bb)^*$,
    получаваме езика от (б) на \textbf{Задача 1}. \\
    (д) Вярно е. Имаме \\
    \begin{center}
        \{$w$ | $w \in L$ $\&$ $w^{rev} \in L\} =$ \\
        \{$w$ | $w \in L$ $\&$ $w \in L^{rev}\} = L \cap L^{rev}$.
    \end{center}
    Но ние вече показахме, че регулярните езици са затворени относно операциите
    обръщане и сечение. \\
    (е) Не е вярно. Въпреки че обединението, приложено краен брой пъти, запазва
    регулярността, ако го приложим върху подходящо безкрайно множество от регулярни
    езици, можем да получим нерегулярен език. Конкретно, да разгледаме в ролята на $R$
    множеството от крайни и следователно регулярни езици $\{\{\epsilon\},\{ab\},\{aabb\},\{aaabbb\},...\}$.
    Очевидно $\bigcup R = \{a^nb^n$ | $n \in \mathbb{N}$\}. \\
    (ж) Вярно е. Този език е точно $\Sigma^*$.

\end{document} 